%% LyX 2.3.5.2 created this file.  For more info, see http://www.lyx.org/.
%% Do not edit unless you really know what you are doing.
\documentclass[spanish]{article}
\usepackage[latin9]{luainputenc}
\usepackage{babel}
\addto\shorthandsspanish{\spanishdeactivate{~<>.}}

\usepackage[unicode=true]
 {hyperref}

\makeatletter

%%%%%%%%%%%%%%%%%%%%%%%%%%%%%% LyX specific LaTeX commands.
%% Because html converters don't know tabularnewline
\providecommand{\tabularnewline}{\\}

\makeatother

\begin{document}
\title{Apuntes del 5/10/2018}
\author{Andrei Leon Salas}
\maketitle

\part{Datos administrativos}

Las clases son los miercoles y viernes de 7:30 am a 9:20 am. Se va
a utilizar teams y zoom para las clases, y la entrega de proyectos,
tareas y apuntes por medio del foro \href{http://ec.tec.ac.cr/index.php/foro/}{http://ec.tec.ac.cr/index.php/foro/}
Si necesita acceso puede solicitarlo a la asistente del profesor Silvia
\href{mailto:}{silmeli97@gmail.com} . Se tiene que postear en el
foro en 4 dias naturales o un poco mas para que quede bien con el
siguiente formato: COM200904-k.pdf siendo k 1 si lo entrego de primero
o 2 si lo posteo de segundo. El sentido de ser apuntador es aprender
bien la materia ya que se tiene que poner mucha atencion en el momento
que el profe habla para hacer buenos apuntes, y adicionalmente extender
la informacion que dio el profe (con graficos, imagenes o informacion
extra). Estos apuntes son revisados por el profe y tiene que venir
bien su informacion para identificarse.

\section*{Formas de contacto}
\begin{itemize}
\item Correo electronico\href{mailto:}{ Torresrojas.cursos.05@gmail.com}:
este correo es exclusivo para este curso, cualquier tarea o proyecto
entregado a otro correo se considera un cero
\item Foro: \href{http://ec.tec.ac.cr/index.php/foro/\%20}{http://ec.tec.ac.cr/index.php/foro/ }:
es el medio oficial y se recomienda hacer las consultas aqui, para
que tambien otros companeros puedan verlo y asi aclarar sus propias
dudas
\end{itemize}

\section*{Evaluacion}
\begin{itemize}
\item Tareas y quices {[}15\%{]}: Se establecio que se haran los miercoles
y no se reponen. Son individuales a menos que el profesor mencione
lo contrario, y se utiliza el correo electronico como medio de entrega
oficial. Todo trabajo escrito se debe utilizar LATEX. Los quices mas
bajos se eliminaran al final del curso, para ayudar al estudiante,
y esta es la razon por la que no se reponen los quices. Se hacen a
MANO, hay que tomarle foto y enviarlo a Riquel \href{mailto:}{driquelme2699@gmail.com}
(NO enviar copia del correo al profesor). La entrega tardia se le
quitaran 3$^{k+1}$ puntos donde k es cada dia de atraso. Las tareas
que sean programadas se haran sobre C en Linux e individuales o en
grupos de 3 dependiendo del enunciado. Si hay una revision y no se
presenta nadie, es tambien un 0 automatico (en grupo tiene que estar
aunque sea uno). Los ensayos tienen las siguientes fechas, tienen
que ser en ingles y en LATEX, un resumen del material y una opinion
personal o conclusion de lo que trataba el material. extension de
4 o 5 paginas. Se entregan directo al profesor \href{mailto:}{ Torresrojas.cursos.05@gmail.com}
\begin{itemize}
\item Ensayo 1 Capitulo 1 Viernes 11 de septiembre
\item Ensayo 2 Capitulo 2 Viernes 18 de septiembre
\item Ensayo 3 Capitulo 3 Viernes 25 de septiembre
\item Ensayo 4 Capitulo 4 Miercoles 4 de octubre
\item Ensayo 5 Capitulo 5 Miercoles 14 de octubre
\end{itemize}
\item Trabajo en clase {[}5\%{]}
\item Examen {[}15\%{]}
\item Examen {[}20\%{]}
\item Examen final {[}10\%{]}
\item Proyectos programadis {[}35\%{]}: todos son hechos en C
\end{itemize}

\section*{Libros de texto }
\begin{itemize}
\item Compiler Construction -- Kenneth C. Louden
\item Compiler Principles, Techniques, and tool -- Alfred V. Aho, Ravi
Sethi, Jeffrey D, Ullman (extra) 
\item Crafting a Compiler with C -- Charles N. Fischer, Richard J.LeBlanc.
Jr(extra) 
\item The Language Instinct- Steven Pinker(extra)
\end{itemize}

\section*{Fraude academico}

Sera fuertemente penalizado cualquier tipo de fraude academico. El
profe enfatizo mucho en esto, y que la escuela esta notando un incremento
de esta mala practica. Hay que usar el sentido comun para saber si
uno esta incurriendo en fraude pero ante la duda el profe dijo que
con gusto el respondia correos con cualquier duda. Algunos ejemplos
de fraude son:
\begin{itemize}
\item Usar el trabajo de otro como propio
\item Conseguir de alguna forma adelantada los examenes o pruebas
\item Usar recursos o cualquier otra cosa que adulteren la evaluacion de
los examenes, proyectos, tareas, etc
\item NO es plagio utilizar informacion del foro de cursos pasados
\end{itemize}

\part{Historia}

\section*{Personas}

\section*{%
\begin{tabular}{|c|c|c|c|c|}
\hline 
Nombre & Nacimiento & Universidad & Contribucion & Premio Turing o tasa de perro\tabularnewline
\hline 
\hline 
John William Mauchly & USA 1907-1980 & Johns Hopkins University & Co-creador de ENIAC & N/A\tabularnewline
\hline 
John Adam Presper Eckert & USA 1919-1995 & Universidad de Pensilvania & Co-creador ENIAC & N/A\tabularnewline
\hline 
John von Neumann & Hungria 1903-1957 & Universidad de Budapest & Arquitectura de computadores, proyecto manhatan, First Draft (Plagio?) & N/A\tabularnewline
\hline 
Sir Maurice Vincent Wilkes & Britanico 1913-2010 & Newcastle University & EDSAC,microprogramacion,subrutinas & 1967\tabularnewline
\hline 
John Backus & USA 1924-2007 & Columbia University & FORTRAN,primer lenguaje de alto nivel,BNF,FP & 1977\tabularnewline
\hline 
\end{tabular}}

\section*{Maquinas}

\subsubsection*{ENIAC}

Programable por medio de cables, era la primer computadora de proposito
general, que se podia cambiar su funcion sin tener que crear la maquina
desde cero. Costo medio millon de dolares (9 millones con la inflacion
actual). Tenia un tiempo de funcionamiento del 50\%, y eran mujeres
quienes principamente se encargaban de toda la programacion de esta
Le decian \guillemotleft Giant Brain\guillemotright{} segun la prensa,
e igual hubo historias de terror de que las computadoras dominarian
el mundo (no tan alejado de la realidad poniendonos filosoficos)

\subsubsection*{EDVAC}

Programable pero en vez de ser cables, se hacia en memoria, muy inovador
en la epoca, tanto que aun ahora se utiliza esta misma idea. Se empezo
a trabajar desde antes que ENIAC estuviera lista, y fue funcional
20 horas diarias con un MTBF (MEAN TIME BETWEEN FAILURES) de 8 horas.
Costo para su momento medio millon de dolares (6643615 millones de
dolares actuales)

\section*{Contribuciones}

\subsubsection*{First draft}

Fue un ensayo sobre el funcionamiento de EDVAC que circulo en 1945,
Se habla mucho de que fue un fraude academico de John von neumann
hacian Mauchly y Presper, junto con este documento se propuso la arquitectura
von Neumann

\subsubsection*{Arquitectura von Neumann}

Prupuesta en 1945 por von Neumann, se comenta que tubo influencia
de Turing y que pudo tener ideas robadas de otros academicos. Basicamente
es la primera arquitectura que integraba la idea del software. Usando
un programa que se lee en la memoria, y ejecutandolo indefinidamente
con un ciclo de fetch. Escencialmente el software

\subsubsection*{Unidad de control }

Es el responsable de ejecutar el ciclo de fetch, tiene de forma simple
los pasos definidos de estar recuperando en memoria las instrucciones,
ejecutarlas, y volver al primer paso indefinidamente, siendo estas
instrucciones posibles de cambiar el estado de la maquina y ejecutar
de forma diferente. No existia antes del EDVAC

\subsubsection*{Diseno de la unidad de control}

Existen dos tipos de disenos, el alambrado que trata de hacer todas
las instrucciones por medios fisicos (compuertas logicas por ejemplo)
y el microprogramado que es tener un peuqeno programa con esas instrucciones
pero por software, no por hardware. La primera es mas eficiente pero
la segunda mas flexible, ya que en la primera no se pueden agregar
instrucciones sin tener que recrear todo, y en la segunda si podemos
agregar sin tener que construir de nuevo el hardware. Cabe destacar
que microprogramada es que ya el set de instrucciones viene hecho
y no se puede cambiar, la microprogramable es la que se puede cambiar

\subsubsection*{Niveles de eficiencia y dificultad}

Como vimos antes, las instrucciones hechas en hardware son mas rapidas
pero menos eficientes. Esta regla se cumple generalmente entre mas
bajo se programa mas eficiente va a ser. Un sistema hecho a partir
de microprogramas va a ser mucho mas veloz que en java, pero va a
ser muy lento crear ese codigo. A esto se le llama brecha semantica.
un sistema asi seria muy eficiente pero poco productivo. 
\begin{itemize}
\item Eficiencia: cantidad de recursos que un sistema en funcionamiento
requiere (el programa que uno hace es eficiente, no la persona)
\item Productividad: la cantidad de lineas de codigo sin errores por unidad
de tiempo al desarrollar un sistema (la persona es quien es productivo,
no el programa hecho)
\end{itemize}
En microprogramacion la eficiencia es alta pero la productividad baja.
En lenguaje maquina la eficiencia baja pero la productividad sube.
Pero en lenguaje ensamblador para una excepcion, la producitivdad
sube pero la eficiencia no baja.

\subsubsection*{Lenguaje ensamblador}

Por no ser tan semantico, es muy tedioso y propenso a errores, por
lo que en 1950 se crea ensamblador, que crea nemonicos y nombres simbolicos
para poder generar lenguaje maquina. Como es equivalente lo generado
por ensamblador a lenguaje maquina es igual de eficiente pero mucho
mas facil de utilizar
\end{document}
