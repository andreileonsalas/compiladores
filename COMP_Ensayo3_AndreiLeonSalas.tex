%% LyX 2.3.5.2 created this file.  For more info, see http://www.lyx.org/.
%% Do not edit unless you really know what you are doing.
\documentclass[spanish,english]{article}
\usepackage[latin9]{luainputenc}
\usepackage{babel}
\addto\shorthandsspanish{\spanishdeactivate{~<>}}

\begin{document}
\title{Chapter 3 The Language Instinct How the Mind Creates Language by Steven
Pinker}
\date{Date 09/25/20}
\author{\selectlanguage{spanish}%
Andrei Le�n Salas\pagebreak}

\maketitle
\tableofcontents{}\pagebreak{}

\section{Summary\index{Summary}}

Author starts putting the year in context, 1984, introducing Newspeak.
Newspeak was the expression of Ingsoc {[}English Socialism{]} and
was intended to replace Oldspeak. Had a vocabulary so constructed
that enabled to give exact and subtle expression to everything a party
member wanted to express. It was done thanks to the creations of new
words, but removing a possible double meaning of the word. For example
the word free, you can use is as the lack of something, but not as
the political meaning as freedom, as the concept itself was removed.
A child growing up using only Newspeak would never know the secondary
meaning of the word, and the concept would remain nameless. Using
this hypothesis Steven point out as how we thing using the words only
as we need to communicate what we think.

Some philosopher even think that as animals can not resonate as they
cannot use the language, so they can not think something like ``will
rain tomorrow''. A extreme view of this is General Semantics, that
attacks the lack of specification in the semantics of the language.
An example is if John went to prison for 40 years, the grammatics
assumes that is the same John, but is not, it should be John 1972
and John 1994. It happens again with the verb to be, as it abstracts
individuals.

But this idea is wrong, people adopted as granted as other myths that
are fake. Even when writing we stop as we cannot completely express
with words what we want to express, or how we could translate if words
were attached to thoughts. Using Monty Python's Flying Circus as an
example, Steven marks that there no scientific evidence that words
equals thinking, instead was an attempt to understant something that
they didn't. Now cognitive scientists know how to think about thinking,
so there are less tries of connect words with thoughts.

Linguistic determinism hypothesis is linked with Edward Sapir and
Benjamin Leww Whof. They argued that non-industrial peoples were not
savages, instead they had a different systems of languages, knowledge
and culture, as complex as ours. Whorf took this further, saying that
we took our own impressions, like a kaleidoscopic where the light
passes as a different opinion on everybody. This idea occurred when
he was working as a fire prevention engineer. People had different
ideas of what an object was. A drum for somebody was empty, meanwhile
for another was full of gasoline vapor. Or a pool of water was really
a basin of decomposing tannery waste. And there was no error in the
example of the empty drum. Was not a mistake of the English grammar
of if it was really empty, it was instead that his eyes fooled him
as he could not see the gasoline vapor filling it.

Another example is the color, Latin lacks generic ``gray'' and brown,
Nabajo combines blue and green into one word and Russian has distinct
words for dark blue and sky blue. 

Whorf also took Hopi concept of time, Hopi contains ``no words, grammatical
forms, constructions or expressions that refer directly to what we
call 'time' or to future, past or to enduring or lasting'', basically
saying that Hopi had no notion of time being another example of how
the point of view changes how people think and speaks.

Eskimo Vocabulary has only 2 words for snow, contrary of the popular
belief that hey have more than 400 words in order to represent snow.
This perception of people believing that they have more than 400 words
was because in 1911 Boas mentioned that they had 4 unrelated word
roots for snow, but as the time passed, and as it became more and
more popular, it began to be exaggerated with each retelling.

Steven continues to show facts of how in the history there has been
a tendency to compare thinking with language, and how it tests falls
as they are not that convincing, and using examples of chapter 2 of
how this was demonstrated

Many creative people insist that when they are inspired, they do not
think in words, instead of images. Famous novelist and writers says
that their act of being the work is not writing, instead imagining
a visual scene and writing around it

Scientists like Tesla ,Clerk Maxwell and even Albert Einstein ideas
also began in a more geometric thinking, not via words. There is an
specific example Steven mentions, where you need to do visual effort
to resolve how many degrees rotated the letter F, as it is not something
you think in words, instead you do visual work to calculate it.

Using all this arguments Steven says that we can get new knowledge
and think without using words or language. 

In essence, to reason is to deduce new pieces of knowledge from old
ones.Being the key the representation that gives the words of it.
Is not the same saying Socrates is a man as saying Mercedes is a man,
it changes all that it representes. Or reading it from left to right
instead of right to left, they have a diferent representaition. And
Steven says that htis representation can also be in others forms,
he used ``ink'' to facilitate it to us, but could also be any physical
medium at all, as long as the patterns are cosistent. In other worlds,
it has to use symbols to represent concepts, and arrangements of symbotls
to represent the logical relations between them.

Finally Steven says that English is unsuited to server as our internal
medium of computation as it has some problems. The first one is ambiguity,
as words can have more than one specific thoguht about it. The second
one is his lack of logical explicitness, as English sentences do not
embody the information that a processor needs to carry out common
sense. He also mentions a third problem, co-reference, as you can
be talking about an specific man, but you can refer him as various
diferents ways ``the tall blond man with one black shoe'', ``the
man'' or only ``him'' we as persons can understand that is refering
to the same person, but the English language does not. The fourth
one is deixis, that some aspects of language only can be interpreted
in the context of the conversation. Being depended of the context
makes also English not good as an internal medium of computation.
A fight one i synonymy, as there many ways to say the same representation.

People do not think in English or Chinese, people think in a language
of thought and we use language to represent it and transmit it to
other people. Finally Steven's predictions for 2050 and Newspeak is
that mental life goes independently of particular languages, and freedom
will be thinkable even if there's no word to represent it. Second
as there are more concepts than words, existing words would gain new
senses. Third children would improve the grammar of it and transform
it (as in chapter 2)

\pagebreak{}

\section{Personal Opinion\index{Opinion}}

It was really difficult to read this one as they had a lot of words
and sentences that were difficult to understand, speaking of representation
haha. I really like how he used chapter 2 to argument his theory in
this chapter. I once took a course of how to read faster and they
taught us that our mind does not read letter by letter, instead we
saw every word and even every sentence as a complete picture (also
that's why sometimes we don't see grammar errors or extra letter on
a sentence), in fact making a representation of an idea as he proposed
in chapter 3. I agree with him in this aspect, but i am not sure about
the computational one, as i could not get it completely. I tried to
relation it with my knowledge of languages in computer engineering,
and is right that we have no ambiguity nor different ways of doing
a move in memory. But we created alias to do it anyways (like c =
c+1 is the same as c++), being the same as he mentioned of how we
try to save in breath, air, time, etc, we did the same in programming
languages. 
\end{document}
