%% LyX 2.3.5.2 created this file.  For more info, see http://www.lyx.org/.
%% Do not edit unless you really know what you are doing.
\documentclass[english]{article}
\usepackage[latin9]{luainputenc}
\usepackage{babel}
\begin{document}
\title{Chapter 1}
\author{Andrei Rolando Leon Salas}
\maketitle

\part{Summary}

In the first chapter the author speaks about how the language is a
wonder of the world. Is really important in order to transmit ideas,
concepts and events among others things. Sounds as incredible as telepathy
or mind control but it is not. Looking the concept of language outside
of the box, instead of taking it for granted, make us amazed of all
the complex things that happens in order to be able to use the language. 

At the beginning he refers that with only making sounds with our mouths,
we can transmit precise ideas to others begins that understand the
language.

He show us, the readers, that he can do that with the writing and
listening skills. He show us 3 different fragments. 

in the first one i imagine as if it were a documentary, even hear
the voice while Reading, how the octopus male tries to mate with a
female of his species, and how the \textquotedblleft act\textquotedblright{}
is done. 

The second one i imagine the specific infomercial, showing me how
to clean a White suit with soda 

The third one i imagine the situation, the people, even the tone of
how Tad said ``I love you''. Is a couple that Dixie thought that
Tad was dead but it was not, but now Dixie is married with Brian and
also that the son is not Brian's son, but Tad

With this examples he explains that he transmitted knowledge to us
without us needing to experience it. Even secrets of the drama All
My Children with millions of people.And even if we need read/write
skills, it let us communicate thought time, space and acquaintanceship. 

He explains that the human being is an ``impressive problem solver
and engineer'' to explain we are a clever species, captured in the
story of the Tower of Babel, which even God felt threatened of us
as with an only language we could improve a lot our intellect as a
specie. 

With communication we can share knowledge, coming from genius, trial
and error or lucky accident, so we can all have that information without
getting it in first hand, empowering the race as a whole.

Language is important that people always use it, even if there's nobody
around, speaking to ourselves, as we can express our self with it.
Aphasia for example makes the family feel like the person is lost
forever.

The book is about human language, but not of his grammatical rules,
instead of a more basic point of view. With the relativity new Cognitive
science, there has been big advances in the language, and we have
become experts in the topic. And he is going to transmit that knowledge
to us via his book.

With the new information we have more understanding of the language
and its role. Some people claim that is the most important cultural
invention, and it is what separates human from other beings. Depending
of how the individual learns a language affects his ability to construct
grammatical sentences, as for example English is sometimes contradictory
and difficult to learn, because it is not as rational as it should
be. Book's objective is to convince us that all of this is wrong because
language is not an artifact, is not something that we created, instead
is part of the biological makeup of our brains.

As complex as language is, we learned to speak it as a child without
conscious effort, in the contrary of others skills that we need to
learn and do it consciously. That's why he claims that we learn by
instinct, a knowledge we have just implanted in our brains, so language
is a unique ability to the human being as breathing for a lot of living
beings.

This concept was first articulated by Darwin at 1871, in The Descent
of Man he did some observations:
\begin{itemize}
\item Language is an art, even is is thought different languages, as a base,
all humans has a tendency to speak. It can not be a true instinct
as every language have to be learned, but differs with ordinary arts,
with babies trying to speak babbling, but no child has the instinct
to bake or write
\item In the modern time, philosopher supposes that language has been slowly
and unconsciously developed by many steps, instead of thinking that
language has been deliberately invented
\item So Language is ``an instinctive tendency to acquire an art''
\end{itemize}
William James extended this, explaining that as animals, we also have
all the instincts as others animals, but our flexible intelligence
comes from the interplay of many instincts competing, and asking why
of any instinctive human act. For example why we smile when we are
happy, instead of scowling. People normally answers Of course we smile,
of course we love, but they don't rationalize about it, it is what
is is.

As mysterious as some animals instincts may be, our instincts are
not that different, as we also feel good when we follow them. We even
speak without realizing what we said and embarrass ourselves because
we just do it, eluding our mental censors. 

Noam Chomsky argued that there were 2 fundamental facts about language:
\begin{enumerate}
\item Every sentence a person said is a new combination of word in the universe,
so it cannot be an action-reaction effect, instead is a recipe or
program that can generate infinite sentences with the words the persona
knows
\item Children develop complex grammars skills rapidly and without formal
instruction, so children should come with an basic Universal Grammar
that helps them learning the language (no matter which one)
\end{enumerate}
Chomsky also said that was curious how everybody have for granted
how the physical structure of an organism is genetically determined,
with all the variations that there are as size, rate of development
and even external factors change human himself. In the same way is
generally assumed that social environment is the dominant factor in
the human cognitive systems, but when seriously investigated are as
intricate as the physical ones. Based on this proposes that language
should be studied as complex as the body. With this Chomsky and other
linguists proposer that there's a Universal Grammar below all others.
Actually there's thousands of more people studying the cognitive system
and, thus, language.

The book is deeply influenced by Chomsky, but is not the story that
Chomsky would said, as Chomsky argue against Darwinian natural selection,
but author thinks that it is a evolutionary adaptation as any other.\newpage{}

\part{Personal opinion}

I found this part interesting, as i never stop myself to think the
language as something rational of instinct. I feel curious how the
author is going to go deeper in the topic and argue why language is
something biological and instinctive instead of something rational.
I first thought as we need to be rational to be able to speak,as one
of the things makes the human being different from animals is his
ability to think. But as i finished the chapter i started to thing,
well what if being rational is also something that started as a instinct. 

To be honest, i need to think more of this topic as i only started
today to thing about this specific topic, and normally i try to learn
from different points of view before having a concussion for myself.
Also i can connect how this concept of language relates to the one
in programming, as right now my opinion is that computer language
is something completely artificial created by us, and cant see how
instinct could related to something so artificial. 

Finally i think Pinker could have a biased opinion as himself wrote
that Chomsky had a power full personality, but not as much as it could,
as he at the end explains that differs with him in various topics
\end{document}
