%% LyX 2.3.5.2 created this file.  For more info, see http://www.lyx.org/.
%% Do not edit unless you really know what you are doing.
\documentclass[spanish,english]{article}
\usepackage[T1]{fontenc}
\usepackage[latin9]{luainputenc}
\usepackage{babel}
\addto\shorthandsspanish{\spanishdeactivate{~<>}}

\begin{document}
\title{Chapter 5 The Language Instinct How the Mind Creates Language by Steven
Pinker}
\date{Date 10/14/20}
\author{\selectlanguage{spanish}%
Andrei Le�n Salas\pagebreak}

\maketitle
\tableofcontents{}\pagebreak{}

\section{Summary\index{Summary}}

Pinker starts referring the dictionary of words, and that it is as
wondrous as the world of syntax. Words are not just a list memorized
in our brain, we have rules in order to create new ones, like the
wug-test. There are 2 places where the tricks of language are used.
In sentences and phases built our of words by syntax rules, and words
by self are build our of smaller pieces, with the rules of ``morphology''.
But English morphology is pathetic compared to other languages, in
English there less words forms compared to Spanish, Italian and classical
Greek. He gives specific examples with the languages already explored
in past chapters, where Kivujo, the most similar one to English, has
the verb \textquotedbl N��k�mly���,'' that has more than half million
combinations, where English has far less. 

English morphology is not bad, has his own pros, creating derivational
words from others, like the suffix able converting verb into an adjective.
Also is easy to compound words, like toothbrush and mouse-eater, making
the words list as extensive as the sentences we can create, infinite.
Like sentences, words are too delicate to be generated with a chaining
device, as example he gives the anti-anti-anti-anti-missiles, where
a chaining device could not put an adjective between anti and missiles,
as it forgets the past state of the word.

Word are consisted by morphemes, pieces that fit together in certain
ways. Like sentences they are like a tree, take the word dogs, divided
between Nstem dog that is the noun steam and Ninflection that is the
noun inflection that makes it plural, in that way we can change the
noun steam with any other noun steam without knowing his meaning,
and make a new word. 

Another rule is join 2 Nstems creating a new noun like Yugoslavia
report. It is not an adjective as grammar teachers teaches, as we
cannot use Yugoslavia as an adjective in a sentence

The adjective stem says that you can join a verb with able to create
an adjective, like crunch-chunchable, also works with er and ness

All of these creating rules are promiscuous, we can exchange it without
any problem, in that way, creating the infinite list of words that
pinker said before.

We can dissect words even more, the smallest piece of a word is called
root, without it, there is no word or meaning. Using the rules that
we check before we can analyze Darwiniannism, where Darwin is the
Nroot, ian is a Nroottaffix creating the noun Darwinian, with the
nstemaffix ism we create Darwinianism, and that noun with the Ninflection
s we create Darwiniansims, 3 levels of a tree is the word Darwiniansims.

There are some exceptions to the rules, like the word electricity.
This is because the original word was created in another language
(Latin in this case) in which followed the rules of their natural
language, so we inherited the words, but not the rules. And even as
we store that word as a tree, we do it with a separation of electric
and -ity, but this separation does not follows any of the English
rules we just mentioned. This exception and the pattern-look's like
rule gave us a whole genre of wordplay, to the point that was parodied.

There are also messy patterns, like irregular plurals and irregular
past-tense forms, this is because years ago, there was a rule that
said that was needed to be replaced one vowel with another to form
the past tense, as our more recent -ed rule. 

Pinker explains how this irregular forms were parodied and studied
for many people, as important as they are, those seems like the epitome
of human eccentricity and quirkiness. This irregular form are explicitly
abolished in rationally designed languages like Esperanto because
of that. Irregular forms are roots, that can be found inside steam,
that can be found inside words, layer after layer, providing an explanation
of why, for example, people says Walkmans instead of Walkmen.

There are 2 explanations on how to book on grammar that are wrong.
One is that we cannot use any new form of irregular form, if new,
has to be regular. Not true as we create words with irregular forms,
their past would be irregular too. The second one is that when a word
acquires a new, non literal sense, it requires a regular form, but
old-mice refute that (regular form should be oil-mouses). 

The head of the word dictates what the would means, like overshoot.
shoot is the head, so overshooting is a kind of shooting.dictating
too their irregular form that is stored with it, making the past tense
of overshoot, overshot instead of overshooted. 

As always there are exceptions, like fly out and Walkmans, they are
headless. And as they are headless, they cannot get their past from
their head. Like low-life, where their plural is not lives, is low-lifes
like any regular word. Or fly out that cannot use flew or flown, it
uses the regular -ed rule flied-out.

Continuing with the tree example, irregular forms are at the bottom
of the tree, where roots and steams are learned directly to the brain
as a mental dictionary. Peter Fordon did an experiment showing how
children's minds seem to be designed with the logic of word structure
built in. Using as an example mice-infested and rat-infested, where
the last one sound wrong. The theory of word structure explains it,
saying that irregular plurals has to be stored as roots or stem in
the mental dictionary, and cannot be generated by a rule. The experiment
was with three to five year old children. With a puppet he asked ``Here
is a monster who likes to eat mud. What do you call him?'' and he
answered mud-eater. Later he did the questions without the answer,and
the children filled the blanks. A monster who eat mice was mice-eater,
but a monster that eat rats was rat-eater instead of rats-eater, suggesting
that the rule takes form in the unconscious mind of the children.
Also he proves that it was not a behavior copied from their parents,
as compounds containing plurals were also produced fine by the children. 

Pinker continues explaining the difference between word and sentence,
that even as word are built of different parts as sentences are, word
does not have a precise meaning by itself compared to sentences. He
says that a word is: ``a linguistic object that, even if built out
of parts by the rules of morphology, behaves as the indivisible, smallest
unit with respect to the rules of syntax---a \textquotedbl syntactic
atom,\textquotedbl{} in atom's original sense of something that cannot
be split'' .

The rules of syntax avoid words to be able to look inside and find
the meaning, meanwhile sentences has this property. 

Another difference with the sense of word, refers to a rote-memorized
chunk: a string of linguistic stuff that is arbitrarily associated
with a particular meaning, retrieved from our metal dictionary. A
listeme, different from a word, can be a tree branch of any size,
as long as it cannot be produced mechanically by rules and has to
be memorized. Even phrase-sized has to be memorized as listemes, like
idioms like kick the bucket or go bananas.

Pinker then focuses on listemes, trying to show that lexicon, is deserving
respect and appreciation.d

is a myth that people knows few words. People say that at average
literate people know a few thousand, Shakespeare 15000, but this is
wrong. Psychologists uses a different method. Start with the largest
unabridged dictionary, draw a sample, for example the third entry
of every eighth left-hand page, and then with each word in the sample,
ask to choose the closest synonym from a set of alternatives. After
that multiply the correct proportion by the size of the dictionary,
and that is the estimate of the person's vocabulary size. Also remember
that Dictionaries are consumer products, and as they are, they inflate
the number of entries.

The most sophisticated estimate comes from William Nagy and Richard
Anderson, they took 227553 different words, from these 45453 are roots
and stems, and the remaining compounds and derivatives. They estimated
that 42080 could be understood in context by someone who knew their
components, in that way, there were already 88533 listeme words. With
that the estimation was that an average American high school graduate
knows 45000 words, three times the Shakespeare number. 

Even children are estimated to know about 13000 words. The brain seems
to have a special capacity to learn new words, like a special storage
just for the mental dictionary as we cannot normally remember that
amount of specific things in our every days. Also we do not learn
just memorizing what the other person just said, it is necessary to
understand the meaning behind the word, the symbol that the word represent.
making the relation between its sound and its meaning utterly arbitrary.
As Shakespeare said, a rose would still smell sweet even with other
name. In that way babies does not expect that cattle mean something
similar to battle, different word, almost the same sound, but completely
different meaning. 

Pinker also mentions ``the scandal of inductions'' which applies
to scientists and children alike: how can they be so successful at
observing a finite set of events and making some correct generalization
about all future events of that kind, rejecting an infinite number
of false generalizations. In programming language we call this the
process of abstraction. Pinker recalls that this could be part of
the common sense, that our mind is designed to find and label words
by itself. We attach words to concepts, and it allows one to share
one's hard-won discoveries and insights about the world. 

\pagebreak{}

\section{Personal Opinion\index{Opinion}}

For the first part, it was really difficult to follow up some of the
words and rules. I took a English course in Fundatec, so they tough
us a lot of specific rules, but decompose the words as trees, and
specially later with all the words parody, make it really difficult
to understand the complete sentence, as those words are not normally
together like that. It is nice to be out of the box once in a while,
but it is also difficult to process it because of that. Like idioms,
i had to google it as the only one i have ever heard was go bananas.

Something that i appreciate of this lecture is that it points out
how much invisible rules are in the English language, making me fell
less bad as sometimes, even with all the courses or speaking daily
English, i have trouble understanding or creating words in English,
even then Pinker emphasizes how the children could understand easily
the sign language, but their parents would never be in the same level
as his daughter as they were already too old. Like in the mice-infested
and rats-infested, pinker says that it sounds wrong, but me as a non
native speaker, both sounds correct. Anna Maria Di Sciullo and Edwin
WIlliams call words, listemes, a unit of a memorized list. If you
change the order of the words, it would not have the same meaning,
but sentences would keep the meaning. 

I agree with Pinker that there are many things that happens almost
unconsciously, and i really like how he explains his points using
children. I always have thought that children are like a blank sheet,
but it is true that if that would be the case, they could not surprise
us with those unique sentences that make us laugh and get surprised.
At the end the common sense and the Darwin evolution looks like just
a point, but it is difficult to further explain something that just
it is what it is. 
\end{document}
