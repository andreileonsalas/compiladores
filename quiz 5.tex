%% LyX 2.3.5.2 created this file.  For more info, see http://www.lyx.org/.
%% Do not edit unless you really know what you are doing.
\documentclass[spanish,english]{article}
\usepackage[latin9]{luainputenc}
\usepackage{babel}
\addto\shorthandsspanish{\spanishdeactivate{~<>}}

\begin{document}
\title{Quiz 5 Que pensamos sobre el fraude acad�mico}
\date{Date 10/14/20}
\author{\selectlanguage{spanish}%
Andrei Le�n Salas\foreignlanguage{english}{\pagebreak Siempre hay
personas que caen en malos actos, sea por vagancia o por estr�s�(como
los ladrones que roban ya sea pq les da pereza trabajar, o por que
no tienen darle de comer a la familia). Siempre he pensando que estos
temas no son ni blancos ni negro, sino algo mas gris. En ese punto
en espec�fico�la persona que hizo fraude me parece que es por vagancia
(copiar de wikipedia es lo mas�vago que uno puede llegar) y no se
le deber�a�perdonar el fraude. Personalmente me enoja ya que yo he
sacado 60's y 70's con mi trabajo peligrando (si saco menos de 80
en la nota final, pierdo el trabajo) y ahora con esto que hizo esta
persona va a ser m�s dif�cil llegar a 80. Yo no soy quien para juzgar,
y todos cometemos errores, pero profesionalmente esa persona no es
alguien�con quien me gustar�a trabajar Mi hermano tambi�n es profesor,
he visto y le he ayudado la cantidad de trabajo que tiene que hacer
para preparar una clase. Lo veo todo el dia respondiendo dudas via
whatsapp. Y la calidad de las presentaciones que hace el profe se
nota que si hace bastante trabajo. Al final si creo que hay que hacer
como en la programaci�n, programar siempre pensando que el usuario
se va a ir por el lado err�neo, y no dando oportunidad de que ponga
una letra donde va un n�mero. En mi trabajo hay varios tipos de certificaciones,
pero las que se hacen a libro abierto, es un examen de selecci�n �nica�que
las preguntas te�ricas son m�nimas, y la mayor�a son de ejecuci�n,
donde se prueban los conocimientos de forma aplicada. Personalmente
creo que as� iban a hacer la mayor�a de los cursos (la c�tedra de
matem�tica lo hace as�,y ha sido bastante bien en realidad, usando
el tool�(GAAP) del tec digital, incluso segun�entendi randomiza�las
preguntas para que a nadie le salga el mismo examen) adjunto las instrucciones
del ultimo quiz por si le interesa verlo.�}\selectlanguage{english}}%
\maketitle

\end{document}
